%
%%%%%%% Chinese Resume/CV LaTeX Template %%%%%%%%%%
%
% This template has been downloaded from:
% 
%
% Author:
% Jianuo Li <lijianuo0411@outlook.com>
% -------------------------------------------------
%环境配置:xelatex格式
%编码条件:UTF-8 Unicode
%--------------------------------------------------
% 配置页面
\documentclass[11pt, a4paper]{cv}% 可选项a4,a5,paper
\geometry{left=1.4cm, top=.8cm, right=1.4cm, bottom=1.8cm, footskip=.5cm}% 尺寸
\fontdir[fonts/] % 页眉页尾设置
% 右侧栏颜色设置
% Awesome Colors: awesome-emerald, awesome-skyblue, awesome-red, awesome-pink
% awesome-nephritis, awesome-concrete, awesome-darknight
% \definecolor{awesome}{HTML}{CA63A8} HTML自定义颜色选项
\colorlet{awesome}{awesome-skyblue}
% 文本颜色设置
%\definecolor{darktext}{HTML}{414141}
% \definecolor{text}{HTML}{333333}
% \definecolor{graytext}{HTML}{5D5D5D}
%\definecolor{lighttext}{HTML}{999999}
% 如果不希望做颜色分区设置false 
\setbool{acvSectionColorHighlight}{true}

\renewcommand{\acvHeaderSocialSep}{\quad\textbar\quad}%个人信息形式 图标+文本

%个人信息
% 斜杠或竖杠的分隔形式: circle|rectangle,edge/noedge,left/right
\photo[rectangle,edge,right]{小废物的照片.jpg}
\name{Jianuo}{Li}
\mobile{(+49) 1522222111}
\email{lijianuo0411@outlook.com}
\github{Lilinowaytogo}
%\position{{\enskip\cdotp\enskip}}
%\address{德国}
%\homepage{}
%\linkedin{}
% \gitlab{}
% \stackoverflow{}{}
% \twitter{}
% \skype{}
% \medium{}
% \googlescholar{}{}
%% \firstname and \lastname will be used
%\extrainfo{申请位置}
%\quote{}
%%%%配置中文环境,下载中文字体
\usepackage{xeCJK}
\setCJKmainfont{NotoSansCJKtc-Regular.otf}
%%%%配置语言环境
\usepackage[english]{babel}
\usepackage[utf8x]{inputenc}
\usepackage[T1]{fontenc}
\usepackage{enumitem}
%--------------------------------------------
\begin{document}
% 信息位置(C: center, L: left, R: right)
\makecvheader[L]
% 页尾三项(<left>, <center>, <right>), 如果不需要,请将其中任何一项留空
\makecvfooter
  {\today}
  {Jianuo Li~~~·~~~CV}
  {\thepage}
%------------------------------------------
% 简历结构(按需求在Resume文件夹中填写信息导入)

%%-------------------------------------------------------------------------------
%	SECTION TITLE
%-------------------------------------------------------------------------------
\cvsection{Summary}


%-------------------------------------------------------------------------------
%	CONTENT
%-------------------------------------------------------------------------------
\begin{cvparagraph}

%---------------------------------------------------------
EXAMPLE:
目前xx从事
有Xx年xx经验
主要从事XXX工作
喜欢XXXX有兴趣为具有挑战性的任务.....,学习新技术..吧啦吧啦
\end{cvparagraph}

%-------------------------------------------------------------------------------
%	SECTION TITLE
%-------------------------------------------------------------------------------
\cvsection{教育经历}


%-------------------------------------------------------------------------------
%	CONTENT
%-------------------------------------------------------------------------------
\begin{cventries}

%---------------------------------------------------------
  \cventry
    {M.Sc. 汽车工程} % Degree
    {XXXUni} % Institution
    {德国} % Location
    {2020 - 04. 2023} % Date(s)
    {
      \begin{cvitems} % Description(s) bullet points
        \item {论文题目:主要课程:}
      \end{cvitems}
    }

%---------------------------------------------------------
\end{cventries}

%-------------------------------------------------------------------------------
%	SECTION TITLE
%-------------------------------------------------------------------------------
\cvsection{项目经历}


%-------------------------------------------------------------------------------
%	CONTENT
%-------------------------------------------------------------------------------
\begin{cventries}

%---------------------------------------------------------
  \cventry
    {担任职责} % Job title
    {项目名称} % Organization
    {城市位置} % Location
    {07. 2022 - 05. 2023} % Date(s)
    {
      \begin{cvitems} % Description(s) of tasks/responsibilities
        \item {项目/实习内容:}
        \item {个人贡献:}
        \item {根据实际情况填写}
        \item {}
        \item {}
        \item {}
      \end{cvitems}
    }

%---------------------------------------------------------





%---------------------------------------------------------
\end{cventries}

%-------------------------------------------------------------------------------
%	SECTION TITLE
%-------------------------------------------------------------------------------
\cvsection{技能}


%-------------------------------------------------------------------------------
%	CONTENT
%-------------------------------------------------------------------------------
\begin{cvskills}

%---------------------------------------------------------
  \cvskill
    {编程语言} % Category
    {Python (Pandas, PyTorch, NumPy, Matplotlib etc.), Matlab} % Skills

%---------------------------------------------------------
  \cvskill
    {自动化办公} % Category
    { \LaTeX (Overleaf),Microsoft Office, Git} % Skills
%---------------------------------------------------------
  \cvskill
    {建模} % Category
    {UG 6.0,Catia,CAE,Protege} % Skills

\end{cvskills}

%-------------------------------------------------------------------------------
%	SECTION TITLE
%-------------------------------------------------------------------------------
\cvsection{语言}
%-------------------------------------------------------------------------------
%	CONTENT
%-------------------------------------------------------------------------------
\begin{cvskills}

%---------------------------------------------------------
  \cvskill
    {中文} % Language
    {母语} % Type

%---------------------------------------------------------
  \cvskill
    {英语} % Language
    {GRE} % Type
%---------------------------------------------------------
  \cvskill
    {德语} % Language
    {流利} % Type
    
%如需添加直接复制模块

\end{cvskills}


%	SECTION TITLE
\cvsection{成就 \& 荣誉}


%	SUBSECTION TITLE
\cvsubsection{国际成就}
%-------------------------------------------------------------------------------
\begin{cvhonors}

%---------------------------------------------------------
  \cvhonor
    {名次/获得称号/入围情况} % Award
    {参与的比赛名称} % Event
    {位置} % Location
    {2019} % Date(s)

%---------------------------------------------------------
  \cvhonor
    {Finalist} % Award
    {参与比赛名称} % Event
    {Las Vegas, U.S.A} % Location
    {2019} % Date(s)


\end{cvhonors}

%可以根据上面的结构再添加一个子章节



%	SECTION TITLE
\cvsection{实习经历}



%	CONTENT
\begin{cventries}

%---------------------------------------------------------
  \cventry
    {} % Role
    {} % Event
    {} % Location
    {} % Date(s)
    {
      \begin{cvitems} % Description(s)
        \item {.}
        \item {.}
      \end{cvitems}
    }

%---------------------------------------------------------

\end{cventries}


%	SECTION TITLE
\cvsection{学术成就/期刊论文}


%	CONTENT
\begin{cventries}

%---------------------------------------------------------
  \cventry
    {} % Role
    {} % Title
    {} % Location
    {T} % Date(s)
    {
      \begin{cvitems} % Description(s)
        \item {.}
      \end{cvitems}
    }

%---------------------------------------------------------
\end{cventries}

%
%	SECTION TITLE
\cvsection{职位/文献成就}


%-------------------------------------------------------------------------------
%	CONTENT
%-------------------------------------------------------------------------------
\begin{cvhonors}

%---------------------------------------------------------
  \cvhonor
    {论文参与者,项目贡献者,期刊撰写者} % Position
    {2019 题目} % Committee
    {城市位置} % Location
    {2018 时间} % Date(s)

%---------------------------------------------------------


\end{cvhonors}

%
%	SECTION TITLE
\cvsection{课外活动}



%	CONTENT
\begin{cventries}

%---------------------------------------------------------
  \cventry
    {} % Affiliation/role
    {} % Organization/group
    {} % Location
    {} % Date(s)
    {
      \begin{cvitems} % Description(s) of experience/contributions/knowledge
        \item {.}
        \item {.}
      \end{cvitems}
    }

%---------------------------------------------------------

\end{cventries}


%-------------------------------------------------------------------------------
\end{document}
